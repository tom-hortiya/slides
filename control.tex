% !TEX root = slides.tex

\section{Control}

\againframe<3>{overview}

\newcommand*{\WidestLeftHandSide}{\small$\accpho(j+1 \mid k)$}
\newcommand*{\WidestRightHandSide}{\small$\mathcal{N}\left(\left[\irr(j \mid k), \, \troom(j \mid k), \, \hroom(j \mid k), \, \xa(j \mid k)\right]^T\right),$}

\begin{frame}<1>[label=mpc1]
    \frametitle{Model Predictive Control Problem I}
    Solve the optimization problem
    {\footnotesize
        \begin{equation*}
            \min_M \, \sum_{j=k}^{K-1} \left\vert \irr(j \mid k) - \irr(j-1 \mid k) \right\vert + V\left(\left[\accirr(K \mid k), \, \accpho(K \mid k), \, \accpow(K \mid k)\right]^T\right)
        \end{equation*}
    }

    \onslide<6->{
        subject to
    }
    {\small
        \begin{align*}
            \onslide<6->{
                \makebox[\widthof{\WidestLeftHandSide}][r]{$\accirr(j+1 \mid k)$} &= \makebox[\widthof{\WidestRightHandSide}][l]{$\accirr(j \mid k) + \tau \cdot \frac{1}{\mathrm{C}_{\irr}} \cdot \irr(j \mid k),$} \\
                \accpho(j+1 \mid k) &= \accpho(j \mid k) + \tau \cdot \frac{1}{\mathrm{C}_{\pho}} \cdot \pho(j \mid k), \\
                \accpow(j+1 \mid k) &= \accpow(j \mid k) + \tau \cdot \frac{1}{\mathrm{C}_{\pow}} \cdot \pow(j \mid k), \\
            }
        \end{align*}
    }
\end{frame}

\begin{frame}<2-4>
    \frametitle{Terminal Cost Function}
    The terminal cost function equals the weighted sum of
    
    \begin{align*}
        &V_{\irr}\left(\accirr(K \mid k)\right) = 
        \begin{cases}
            \optaccirr - \accirr(K \mid k) & \text{if } \accirr(K \mid k) \leq \optaccirr \\
            0 & \text{else}
        \end{cases} \\[0.5cm]
        \onslide<3->{
            &V_{\pho}\left(\accpho(K \mid k)\right) = - \accpho(K \mid k)\\[0.5cm]
        }
        \onslide<4>{
            &V_{\pow}\left(\accpow(K \mid k)\right) = \accpow(K \mid k)
        }
    \end{align*}

\end{frame}

\againframe<5-6>{mpc1}

\begin{frame}<6-7>[label=mpc2]
    \frametitle{Model Predictive Control Problem II}
    {\small
        \begin{align*}
            \onslide<6->{
                \makebox[\widthof{\WidestLeftHandSide}][r]{$\irr(j \mid k)$} &= \makebox[\widthof{\WidestRightHandSide}][l]{$\irrnat(j \mid k) + \irrart(j \mid k),$} \\
            }
            \onslide<7->{
                \pho(j \mid k) &= \mathcal{N}\left(\left[\irr(j \mid k), \, \troom(j \mid k), \, \hroom(j \mid k), \, \xa(j \mid k)\right]^T\right), \\
            }
            \onslide<11->{
                \pow(j \mid k) &= \frac{\maxpow}{\maxirrart} \cdot \irrart(j \mid k), \\
            }
            \onslide<12->{
                \irrart(j \mid k) &\geq 0, \\
                \irrart(j \mid k) &\leq \maxirrart, \\
            }
            \onslide<13->{\accpow(K \mid k) &\leq \maxaccpow}
        \end{align*}
    }

    \onslide<14>{
        for $j = k , \ldots , K-1$ as well as with $\accirr(k \mid k) = \accirr(k)$, $\accpho(k \mid k) = \accpho(k)$ and $\accpow(k \mid k) = \accpow(k)$.
    }
\end{frame}

\begin{frame}<8>
    \frametitle{Neural Network Formulation I}

    A three-layer neural network can be represented by constructing mixed-integer programming formulations for its units.

    \vspace{0.5cm}

    \begin{figure}
        \centering
        \resizebox{8cm}{!}{
            \begin{tikzpicture}[node distance=1em and 2em, >=latex, font=\footnotesize]
                \node[data] (x1) {$x_1$};
                \node[data, below=of x1] (x2) {$x_2$};
                \node[circle, below=of x2] (dots) {\rotatebox{90}{$\cdots$}};
                \node[data, below=of dots] (xn) {$x_n$};

                \node (helper) at ($(x2)!0.5!(dots)$) {};

                \node[operation, right=of helper, xshift=2em] (sum) {$\sum_{i=1}^n w_i \cdot x_i + b$};
                \node[data, right=of sum] (y) {$y$};
                \node[operation, right=of y] (max) {$\max\{0,y\}$};
                \node[data, right=of max] (z) {$z$};

                \node[above=of sum, yshift=1em] (b) {$b$};  

                \draw [->] (x1) -- node[near start, above=0.25em] {$w_1$} ($(sum.west) + (0,0.2)$);
                \draw [->] (x2) -- node[near start, above] {$w_2$} (sum.west);
                \draw [->] (xn) -- node[near start, above=0.25em] {$w_n$} ($(sum.west) - (0,0.2)$);

                \draw [->] (b) -- (sum);

                \draw [->] (sum) -- (y);

                \draw [->] (y) -- (max);

                \draw [->] (max) -- (z);
            \end{tikzpicture}
        }
        \caption{Rectified linear unit.}
    \end{figure}
\end{frame}

\begin{frame}<9>
    \frametitle{Neural Network Formulation II}
    
    A mixed-integer programming formulation for the rectifier activation function can be constructed as 
    \begin{align*}
        z &\geq 0, \\
        z &\geq y, \\
        z &\leq y - M^- \cdot (1-\delta), \\
        z &\leq M^+ \cdot \delta
    \end{align*}

    where $\delta \in \{0, \, 1\}$ is a binary variable and $M^-, M^+ \in \mathbb{R}$ are constants \cite{Anderson2020}.

\end{frame}

\againframe<10->{mpc2}